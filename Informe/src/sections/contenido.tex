\section{Introduccion}
Este proyecto tiene como objetivo crear un programa que lea una secuencia de ADN (formada por letras A, C, G y T) y busque genes cortos dentro de ella. 
Para ello se utilizaran diversas estructuras de datos. Se implementara un arbol 4-ario (tipo trie) para guardar las combinaciones posibles 
de genes y listas enlazadas para guardar las posiciones donde aparecen. 
Se implementara la deteccion de comandos, entre ellos estaran:
\begin{itemize}
    \item  \textbf{bio start m:} crea un arbol 4-ario con genes de tamaño m
    \item  \textbf{bio read archivo.txt:} lee la secuencia de ADN desde el archivo archivo.txt
    \item  \textbf{bio search XX:} busca el gen XX en la secuencia
    \item  \textbf{bio max:} muestra el gen mas repetido en la secuencia
    \item  \textbf{bio min:} muestra el gen menos repetido en la secuencia
    \item  \textbf{bio all:} muestra todos los gen que tienen frecuencia > 0
    \item  \textbf{bio exit:} libera memoria y sale del programa
\end{itemize}
El proyecto se desarrolla en el lenguaje de programacion C.

\newpage
\section{Objetivos}
\begin{itemize}
    \item Implementar un sistema de búsqueda de patrones utilizando árboles tries.
    \item Implementar y manipular estructuras de datos abstractas como árboles y listas enlazadas para almacenar y gestionar los datos manipulados.
    \item Desarrollar habilidades en programación en lenguaje C, centrándose en el manejo de memoria, punteros y eficiencia algorítmica.
\end{itemize}

\newpage
\section{Estructuras de datos}
Para la implementación del proyecto se utilizaron las siguientes estructuras de datos:
\begin{itemize}
    \item Lista enlazada (Linked List)
    \item Arbol 4-ario (Trie)
\end{itemize}
    \subsection{Lista enlazada}
    Una \textbf{lista enlazada} es una estructura de datos dinámica que consiste en una serie de elementos llamados nodos, donde cada nodo contiene un dato y una referencia 
    al siguiente nodo en la secuencia. A diferencia de un array, las listas enlazadas permiten un crecimiento dinámico sin un tamaño fijo y facilitan la inserción 
    y eliminación de elementos, ya que solo requieren actualizar referencias.\\

    Sin embargo, las listas enlazadas presentan tanto ventajas como desventajas frente a los arreglos:
    \begin{itemize}
        \item \textbf{Ventajas:} Inserciones y eliminaciones en cualquier posición sin necesidad de mover todos los elementos. Crecimiento dinámico sin tamaño predefinido.
        \item \textbf{Desventajas:} La búsqueda de un dato específico requiere recorrer la lista nodo por nodo ($O(N)$), mientras que en un arreglo, si se conoce el índice, se puede acceder directamente en $O(1)$.
    \end{itemize}
    Para comprender mejor una lista enlazada es importante saber que es un nodo.
        \subsubsection{Nodo}
        En informática, un nodo es un elemento individual de una estructura de datos que almacena información y, en muchos casos, tiene referencias o enlaces a otros nodos. 
        Los nodos son la base de muchas estructuras de datos dinámicas, como listas enlazadas, árboles y grafos.\\
        Un nodo tiene dos componentes principales:
        \begin{itemize}
            \item  \textbf{Dato}: La información que el nodo almacena, que puede ser de cualquier tipo de dato, como un número, una cadena de texto o un objeto más complejo.
            \item  \textbf{Referencia}:Un puntero o referencia al siguiente nodo en la lista
        \end{itemize}
        En la siguiente imagen se puede apreciar de manera grafica que es un nodo y tambien un extracto de codigo en C que representa un nodo de una lista enlazada.
        \begin{figure}[H]
            \centering % Para que aparezca centrada
            \includegraphics[width=0.5\textwidth]{./src/images/Nodo.png} 
            \caption{Nodo} 
            \label{fig: nodo} 
        \end{figure}

        \begin{lstlisting}[style=CodeStyle, language=C, caption={Creacion de Nodo}, label={lst:codigo}]
        /*Definición del nodo*/
        typedef struct Nodo {
            int dato;               /*Valor almacenado*/
            struct Nodo* siguiente; /*Puntero al siguiente nodo*/
        } Nodo;

        /*Creación de un nuevo nodo*/
        Nodo* crearNodo(int valor) {
            Nodo* nuevo = (Nodo*) malloc(sizeof(Nodo));
            nuevo->dato = valor;
            nuevo->siguiente = NULL;
            return nuevo;
        }
        \end{lstlisting}

    Las listas enlazadas tienen un nodo inicial llamado \textbf{head} el cual apunta al primer elemento de la lista. Si se llega a un nodo que no apunta a ningún otro, 
    se dice que dicho nodo apunta a \textbf{null} o final de la lista.\\
    En la siguiente imagen se puede apreciar de manera grafica que es una lista enlazada y su implementacion en C
    \begin{figure}[H]
        \centering % Para que aparezca centrada
        \includegraphics[width=0.9\textwidth]{./src/images/ListaEnlazada.png} 
        \caption{Lista enlazada} 
        \label{fig: lista} 
    \end{figure}

    \begin{lstlisting}[style=CodeStyle, language=C, caption={Estructura lista enlazada}, label={lst:codigo}]
    /*Inserta un nodo al final de la lista*/ 
    void insertarFinal(Nodo** cabeza, int valor) {
        Nodo* nuevo = crearNodo(valor);
        if (*cabeza == NULL) {
            *cabeza = nuevo;
            return;
        }
        Nodo* actual = *cabeza;
        while (actual->siguiente != NULL) {
            actual = actual->siguiente;
        }
        actual->siguiente = nuevo;
    }
    \end{lstlisting}

    La lista enlazada, se utiliza dentro del arbol 4-ario para almacenar las posiciones donde aparecen los genes.
    En el nodo hoja del Trie (aquel que corresponde al final de un gen de longitud m), se almacena un puntero a una lista \verb|(ListaInt* lista_posiciones)|. 
    Cada nodo de esta lista (ListaInt) guarda un entero: la posición (índice 0-basado) donde ese gen comienza en la secuencia S. Es decir cada vez que el programa detecta 
    un gen (una secuencia de longitud m) dentro de la secuencia principal de ADN, se almacena su posición inicial (el índice donde empieza).  

    Como un mismo gen puede aparecer muchas veces en distintas partes de la secuencia, se necesita una estructura que permita agregar fácilmente nuevas posiciones
    sin un tamaño fijo ni un límite predefinido. Aqui es donde se utiliza la \textbf{lista enlazada}.
    Tambien permite que el programa no solo diga si un gen existe, sino también cuántas veces (bio search, bio max, bio min) y dónde comienza cada aparición.
    
    \subsection{Arboles}
    Los Árboles son las estructuras de datos mas utilizadas, pero también una de las mas complejas, Los Árboles se caracterizan por almacenar sus nodos en 
    forma jerárquica y no en forma lineal como las Listas enlazadas, Colas,Pilas,etc.
     \begin{figure}[H]
        \centering % Para que aparezca centrada
        \includegraphics[width=0.4\textwidth]{./src/images/ArbolvsLista.png} 
        \caption{Visualización de árbol vs lista} 
        \label{fig: lista vs arbol} 
    \end{figure}

    Para comprender que es un arbol, es importante entender como esta estructurado y cuales son sus componentes. Algunos de ellos son:
    \begin{itemize}
        \item \textbf{Nodo:} Se le llama nodo a cada elemento que contiene un arbol.
        \item \textbf{Raiz:} Es el nodo principal del arbol, es el primer nodo del arbol, solo puede haber un nodo como raiz.
        \item \textbf{Padre:} Este termino representa a cualquier nodo que tiene al menos un hijo.
        \item \textbf{Hijo:} Los hijos son los nodos que tienen un padre.
        \item \textbf{Nodo hermano:} Los nodos hermanos son aquellos que comparten el mismo padre.
        \item \textbf{Hoja:} Son los nodos que no tienen hijos, por lo tanto siempre se encuentran en los extremos de un arbol.
    \end{itemize}
    Puede ser algo dificil de comprender, por lo que a continuacion se muestra una imagen que ejemplifica los conceptos antes mencionados.
     \begin{figure}[H]
        \centering % Para que aparezca centrada
        \includegraphics[width=0.5\textwidth]{./src/images/Arbol.png} 
        \caption{Componentes de un arbol} 
        \label{fig: componentes arbol}
    \end{figure}
    Los árboles a demas de nodos tienen otras propiedades importantes como:
    \begin{itemize}
        \item \textbf{Nivel:} El nivel es cada generacion dentro del arbol, por ejemplo: si se tiene un nodo hoja y se le agrega un hijo, entonces el nodo hoja se convierte
        en un nodo rama, pero a demas el arbol aumenta un nivel mas.
        \item \textbf{Altura:} La altura de un arbol es la cantidad de niveles que tiene el arbol.
        \begin{figure}[H]
            \centering % Para que aparezca centrada
            \includegraphics[width=0.35\textwidth]{./src/images/nivelyaltura.png} 
            \caption{Nivel y altura de un arbol} 
            \label{fig: nivel y altura de un arbol} 
        \end{figure}
        \item \textbf{Peso:} El peso corresponde a la cantidad de nodos que tiene un arbol, es un concepto muy importante ya que permite tener una idea de que tan grande es 
        el arbol y el tamaño de memoria que usara en ejecucion.
        \begin{figure}[H]
            \centering % Para que aparezca centrada
            \includegraphics[width=0.35\textwidth]{./src/images/peso.png} 
            \caption{Peso de un arbol} 
            \label{fig: peso de un arbol} 
        \end{figure}
    \end{itemize}
    A continuacion un estracto de codigo en C de un arbol binario:
    \begin{lstlisting}[style=CodeStyle, language=C, caption={Arbol binario}, label={lst:codigo}]
    /*Estructura para un Nodo de Árbol Binario (2 hijos)*/
    typedef struct NodoArbol {
        int dato; /*El valor que almacena el nodo*/
        /*Punteros a los hijos (recursivos)*/
        struct NodoArbol* izquierda; /*Puntero al subárbol izquierdo*/
        struct NodoArbol* derecha;  /*Puntero al subárbol derecho*/
    } NodoArbol;
    \end{lstlisting}
    \subsubsection{Arbol 4-ario (Trie)}
    Existen muchas formas de implementar un arbol, por ejemplo un arbol binario, pero en este proyecto se utilizara un arbol 4-ario (tipo trie). 
    Este tipo de arbol tiene 4 hijos posibles, uno por cada letra del ADN (A, C, G y T). Los tries se usan comunmente en sistemas de busqueda de 
    patrones porque permiten recorrer y almacenar grandes cantidades de datos de forma ordenada y eficiente. Gracias a su estructura, es posible encontrar 
    patrones dentro de una secuencia de manera rapida sin necesidad de comparar cada combinacion manualmente.\\
    El arbol 4-ario se puede representar mediante la siguiente imagen:
    \begin{figure}[H]
        \centering % Para que aparezca centrada
        \includegraphics[width=0.5\textwidth]{./src/images/arbol4.png} 
        \caption{Arbol 4-ario} 
        \label{fig: arbol 4-ario}
    \end{figure}
\newpage
\section{Implementacion de comandos}
En este proyecto se implementaron principalmente siete comandos para interactuar con el programa. 
Estos comandos permiten al usuario iniciar el arbol, leer la secuencia de ADN, buscar genes,etc.
    \subsection{bio start m}
    Este comando crea un arbol de orden m, donde m es el tamaño de la secuencia de ADN.
        \begin{lstlisting}[style=CodeStyle, language=C, caption={Implementacion bio start m}, label={lst:codigo}]
            Nodo* bio_start(int m) {
            if (m <= 0) {
                printf("Error: El tamano del gen (m) debe ser mayor a 0.\n");
                return NULL;
            }
            if (TRIE_ROOT != NULL) {
                bio_exit();
            }
            GEN_SIZE = m;
            TRIE_ROOT = generar_arbol_recursivo(0, m);

            if (TRIE_ROOT != NULL) {
                printf("Arbol creado con tamano %d\n", m);
            } else {
                printf("Error al crear el arbol.\n");
            }
            return TRIE_ROOT;
        }
        \end{lstlisting}

    \subsection{bio read archivo.txt}
    Este comando lee la secuencia de ADN desde el archivo archivo.txt.
        \begin{lstlisting}[style=CodeStyle, language=C, caption={Fragmento clave read archivo.txt}, label={lst:codigo}]
            /*Logica para leer toda la secuencia S*/
            fseek(file, 0, SEEK_END);
            long n_size = ftell(file);
            fseek(file, 0, SEEK_SET);

            char* secuencia_S = (char*)malloc(n_size + 1);
            /*Verificación de malloc*/
            size_t read_len = fread(secuencia_S, 1, n_size, file);
            fclose(file);
        \end{lstlisting}
        \textbf{Aginacion dinamica:}Se utiliza fseek y ftell para determinar el tamaño exacto del archivo (n) y luego se asigna dinámicamente el buffer \verb|secuencia_S| con malloc.
        Esto garantiza que la secuencia genética completa se cargue en la memoria (heap) de un solo golpe con fread, mejorando la eficiencia de lectura del archivo.

    \subsection{bio search XX}
    La función \verb|bio_search| implementa el núcleo del sistema de búsqueda. Su objetivo es encontrar la frecuencia y todas las posiciones de inicio de un gen específico 
    (de longitud m) dentro de la secuencia genética S previamente cargada en el Árbol Trie
        \begin{lstlisting}[style=CodeStyle, language=C, caption={Fragmento clave bio search XX}, label={lst:codigo}]
            /** Verificar que el arbol esta inicializado **/
            if (TRIE_ROOT == NULL || GEN_SIZE == 0) return -1;
            /** Validar que el gen tenga el largo correcto **/
            if ((int)strlen(gen_G) != GEN_SIZE) return -1;
            /** Recorrer el trie segun cada caracter del gen **/
            for (int i = 0; i < GEN_SIZE; i++) {
                int index = get_base_index(gen_G[i]);
                if (index == -1) return -1;     /** caracter invalido **/
                actual = actual->hijos[index];  /** avanzar en el arbol **/
            }
            /** Contar cuantas veces aparece el gen **/
            int frecuencia = contar_posiciones(actual->lista_posiciones);
            if (frecuencia == 0) return -1;
            /** Imprimir el gen, su frecuencia y posiciones **/
            printf("%s %d", gen_G, frecuencia);
            imprimir_posiciones(actual->lista_posiciones);
        \end{lstlisting}

    \subsection{bio max, bio min, bio all}
        Estos comandos permiten obtener el gen mas repetido (bio max), el gen menos repetido (bio min) y todos los gen que tienen frecuencia > 0 (bio all).
        A continuacion extractos de cada uno de estos comandos
        \subsubsection{bio max}
            Busca dentro del árbol el gen con mayor frecuencia. Se realiza un recorrido completo (DFS) y en cada 
            hoja se compara la frecuencia actual con la maxima encontrada hasta el momento.
                \begin{lstlisting}[style=CodeStyle, language=C, caption={Funcion bio max}, label={lst:codigo}]
                void bio_max() {
                    if (TRIE_ROOT == NULL || GEN_SIZE == 0) { printf("Error: Arbol no cargado.\n"); return; }
                    char* gen_buffer = (char*)malloc(GEN_SIZE + 1);
                    if (gen_buffer == NULL) { perror("Error de asignacion de memoria"); return; }
                    int max_freq = 0; 
                    recorrido_extremos(TRIE_ROOT, gen_buffer, 0, 1, &max_freq);
                    if (max_freq == 0) { printf("No hay genes presentes.\n"); free(gen_buffer); return; }
                    imprimir_recorrido(TRIE_ROOT, gen_buffer, 0, max_freq);
                    free(gen_buffer);
                }
                \end{lstlisting}
        \subsubsection{bio min}
            Similar a bio max, pero busca el gene con frecuencia minima > 0. Ignora hojas sin apariciones. Comparte la misma estrategia de recorrido pero con la condicion inversa.
            \begin{lstlisting}[style=CodeStyle, language=C, caption={Funcion bio min}, label={lst:codigo}]
            void bio_min() {
                if (TRIE_ROOT == NULL || GEN_SIZE == 0) { printf("Error: Arbol no cargado.\n"); return; 
                char* gen_buffer = (char*)malloc(GEN_SIZE + 1);
                if (gen_buffer == NULL) { perror("Error de asignacion de memoria"); return; }
                int min_freq = INT_MAX;
                recorrido_extremos(TRIE_ROOT, gen_buffer, 0, 0, &min_freq);
                if (min_freq == INT_MAX) { printf("No hay genes presentes.\n"); free(gen_buffer); return; }
                imprimir_recorrido(TRIE_ROOT, gen_buffer, 0, min_freq);
                free(gen_buffer);
            }
            \end{lstlisting}
        \subsubsection{bio all}
            Recorre todas las hojas del arbol y muestra solo aquellas cuyo conteo de apariciones es mayor a cero. Es una enumeración completa de 
            todos los genes presentes en los datos cargados.
            \begin{lstlisting}[style=CodeStyle, language=C, caption={Funcion bio all}, label={lst:codigo}]
            void bio_all() {
                if (TRIE_ROOT == NULL || GEN_SIZE == 0) { printf("Error: Arbol no cargado.\n"); return; }
                char* gen_buffer = (char*)malloc(GEN_SIZE + 1);
                if (gen_buffer == NULL) { perror("Error de asignacion de memoria"); return; }
                /*Funcion recursiva interna para ALL (imprime todos los que tengan frecuencia > 0)*/ 
                void bio_all_recursivo(Nodo* nodo, char* buffer, int nivel) {
                    if (nodo == NULL) return;
                    if (nivel == GEN_SIZE) {
                        int frecuencia = contar_posiciones(nodo->lista_posiciones);
                        if (frecuencia > 0) {
                            buffer[GEN_SIZE] = '\0';
                            printf("%s %d", buffer, frecuencia);
                            imprimir_posiciones(nodo->lista_posiciones);
                            printf("\n");
                        }
                        return;
                    }
                    const char bases[] = {'A', 'C', 'G', 'T'};
                    for (int i = 0; i < 4; i++) {
                        buffer[nivel] = bases[i];
                        bio_all_recursivo(nodo->hijos[i], buffer, nivel + 1);
                    }
                }
                bio_all_recursivo(TRIE_ROOT, gen_buffer, 0);
                free(gen_buffer);
            }
            \end{lstlisting}
            El arreglo \verb|gen_buffer| se utiliza para reconstruir el gen letra por letra durante el recorrido del árbol. Como cada nivel del Trie representa una base del 
            ADN, el buffer permite almacenar la 'ruta' tomada para llegar a cada hoja, reconstruyendo asi el gen completo sin necesidad de almacenarlo textualmente en cada nodo 
            (lo cual ahorra memoria).
        \subsubsection*{Complejidad computacional}
            \begin{itemize}
            \item El recorrido completo del Trie tiene complejidad $O(4^m)$ donde $m = GEN_SIZE$ (longitud del gen). 
            \item Esto es eficiente porque el tamaño del alfabeto es fijo (A,C,G,T) y m es pequeño y constante en el contexto biologico.
            \item Las funciones bio max, bio min y bio all tienen la misma complejidad porque requieren recorrer todas las hojas.
            \end{itemize}
        \subsection{bio exit}
            void bio_exit() {
            liberar_arbol(TRIE_ROOT);
            TRIE_ROOT = NULL;
            GEN_SIZE = 0;
            printf("Limpiando cache y salendo del programa...\n");
            }
            \end{lstlisting}
        \subsection{help}
        Este comando es un adicional, ya que no estaba en los requerimientos iniciales, pero se considero importante para el usuario. Ya que al iniciar el programa no 
        hay nada que indique como funciona si el usuario no tiene conocimientos previos. Para todo lo relacionado con la salida por terminal se utilizo definicion de macros
        con el fin de tener una interfaz mas atractiva para el usuario.\\
        A continuacion la salida por terminal de la funcion \verb|help|:
        \begin{figure}[H]
        ]
            \centering % Para que aparezca centrada
            \includegraphics[width=0.6\textwidth]{./src/images/help.png} 
            \caption{Salida por terminal del comando help} 
            \label{fig: help}
        \end{figure}
\newpage
\section{Diseño del programa (arquitectura general)}
Al ejecutar el programa pasa por los siguientes pasos:
\begin{itemize}
    \item bio start 4: Crea un arbol 4-ario con genes de tamaño 4.
    \item bio read adn.txt: Lee la secuencia de ADN desde el archivo adn.txt.
    \item bio search XX: Busca el gen XX en la secuencia.
    \item bio maz/min/all: recorren todo el arbol y muestran los genes que cumplen con la condición.
    \item bio exit: libera memoria y sale del programa.
\end{itemize}
Al ejecutar se tiene la siguiente salida por terminal:
\begin{figure}[H]
    \centering % Para que aparezca centrada
    \includegraphics[width=0.7\textwidth]{./src/images/salida.png} 
    \caption{Salida por terminal del programa} 
    \label{fig: salida por terminal}
\end{figure}

\newpage
\section{Manejo de memoria}
El programa hace un uso intensivo de memoria dinamica, ya que tanto el arbol (trie) como las listas enlazadas se construyen en tiempo de ejecucion a 
partir de la secuencia de ADN cargada. Para ello se utilizan los siguientes mecanismos:
\begin{itemize}
    \item \textbf{malloc:} Se emplea para reservar memoria de forma dinamica para cada nodo del arbol y para cada nodo de las listas enlazadas donde se almacenan las posiciones de aparicion de cada gen.
    \item \textbf{free:} Se utiliza para liberar toda la memoria asignada una vez que el programa finaliza o cuando se ejecuta el comando \texttt{bio exit}. Esto evita fugas de memoria (memory leaks).
    \item \textbf{Estructuras dinamicas:} 
    El arbol se implementa como un trie 4-ario, donde cada nodo contiene un arreglo de cuatro punteros (\texttt{A, C, G, T}). En las hojas del arbol, cada gen almacena una lista enlazada con todas las posiciones donde aparece en la secuencia.
\end{itemize}
Cada nodo del arbol se reserva mediante \texttt{malloc}, incluyendo su arreglo de punteros a hijos. De forma similar, cada posicion encontrada en la secuencia se 
inserta en una lista enlazada, cuyos nodos tambien son reservados dinamicamente.
Para asegurar una correcta administracion de memoria, el programa realiza un recorrido postorden del arbol al momento de liberar los recursos. En este proceso:
\begin{enumerate}
    \item Se liberan primero todas las listas enlazadas asociadas a cada hoja.
    \item Luego se libera cada nodo del arbol recursivamente.
    \item Finalmente, se libera la memoria utilizada por la secuencia principal y demas buffers auxiliares.
\end{enumerate}
Esto garantiza que ninguna referencia quede perdida y que toda la memoria solicitada al sistema sea correctamente devuelta.
\newpage
\section{Conclusion}
El desarrollo de este proyecto permitio implementar un sistema eficiente para el analisis de secuencias de ADN mediante el uso de estructuras de datos dinamicas. 
El arbol 4-ario (trie) demostro ser especialmente adecuado para almacenar y buscar genes de longitud fija, permitiendo realizar consultas en tiempo O(m) sin depender del 
tamano total de la secuencia. Complementariamente, las listas enlazadas utilizadas en cada nodo hoja permitieron registrar todas las posiciones de aparicion de cada gen, 
proporcionando informacion completa y facilmente accesible.
Ademas, la gestion dinamica de memoria mediante \texttt{malloc} y \texttt{free} aseguro un uso responsable de los recursos, evitando fugas y permitiendo la liberacion 
ordenada de todas las estructuras al finalizar la ejecucion del programa.
En conjunto, el proyecto integra de manera efectiva conceptos fundamentales como arboles, listas enlazadas, manejo de memoria y diseño modular. 
El resultado es una herramienta funcional que permite procesar y consultar secuencias geneticas de forma rapida, clara y eficiente.
